% Version 1: Bernd
% Version 2: Omar
\documentclass[usenatbib,usegraphicx,useAMS,onecolumn]{mn2e}
\usepackage[T1]{fontenc}
%\usepackage{txfonts}
\usepackage{multirow}
\usepackage[dvipsnames]{xcolor}
\usepackage{times}
\usepackage{amssymb}
\usepackage{listings}
\usepackage{enumitem}
\newcommand{\COzero}{${}^{12}\mathrm{CO}(1-0)$}
\newcommand{\COtwo}{${}^{12}\mathrm{CO}(2-1)$}
\newcommand{\Fe}{\textsc{feii}}
\newcommand{\Ox}{\textsc{[oiii]}}
\newcommand{\Ni}{\textsc{[nii]}}
\newcommand{\Su}{\textsc{[sii]}}
\newcommand{\Hb}{\mbox{H$\beta$}}
\newcommand{\Ha}{\mbox{H$\alpha$}}
\newcommand{\HII}{\textsc{hii}}
\newcommand{\QDeb}{\textsc{qdeblend${}^{\mathrm{3D}}$}}
\newcommand{\changed}[1]{\textbf{#1}}
%\newcommand{\changed}[1]{#1}
\newcommand{\pasp}{PASP}
\newcommand{\aj}{AJ}
\newcommand{\apj}{ApJ}
\newcommand{\apjs}{ApJS}
\newcommand{\aap}{A\&A}
\newcommand{\aaps}{A\&AS}
\newcommand{\araa}{ARA\&A}
\newcommand{\mnras}{MNRAS}
\newcommand{\apjl}{ApJL}
\lstset{basicstyle=\footnotesize\ttfamily, breaklines=true}
\newcommand{\codeline}[1]{\lstinline|#1|}
\newcommand{\paramdef}[1]{\textcolor{red}{\codeline{#1}}}
\newcommand{\tblcol}[1]{\textcolor{blue}{\codeline{#1}}}
\newcommand{\fname}[1]{\textcolor{magenta}{\codeline{#1}}}
\newcommand{\headerkw}[1]{\textcolor{ForestGreen}{\codeline{#1}}}
\newcommand{\headerval}[1]{\textcolor{YellowGreen}{\codeline{#1}}}

\newcommand{\myemail}{bhuseman@eso.org}
%\bibpunct{(}{)}{;}{a}{}{,}

\definecolor{deepblue}{rgb}{0,0,0.5}
\definecolor{deepred}{rgb}{0.6,0,0}
\definecolor{deepgreen}{rgb}{0,0.5,0}

\begin{document}

\title{PyParadise User Manual}

\author[Husemann et al.]{B.~Husemann$^{1}$\thanks{ESO fellow, bhuseman@eso.org}, O. S. Choudhury$^{2}$, C. J. Walcher$^{2}$\newauthor\\
$^1$European Southern Observatory, Karl-Schwarzschild-Str. 2, 85748 Garching b. M\"unchen, Germany\\
$^2$Leibniz-Institut f\"ur Astrophysik Potsdam (AIP), An der Sternwarte 16, 14482 Potsdam, Germany\\
}
\maketitle
\begin{abstract}

\end{abstract}
%\begin{keywords}
%\end{keywords}

\section{Introduction}
\label{sec:introduction}
There are a number of codes available for deriving stellar population properties from full-spectral fitting and/or emission lines, each with various advantages and disadvantages.
In general, there are three different approaches to obtain derive the star-formation histories from stellar populations: codes that parametrise the star-formation histories, codes that regularise the star-formation histories and codes that do neither.
For a full discussion of different codes and analysis, we refer to {Walcher2011}.
List of codes: \textsc{STARLIGHT}, \textsc{STECKMAP}, \textsc{ULYSS}, \textsc{pPXF}, \textsc{FIT3D}.

\textsc{STECMAP} is developed by {Ocvirk2006a} to obtain the star-formation histories with fixed kinematics.
There is an extension named \textsc{STECKMAP} {Ocvirk2006b}, which can also fit the kinematics.
The codes use a penalised maximum-likelihood analysis to smooth the star-formation histories and applies Monte-Carlo simulations to derive uncertainties.

\textsc{pPXF} is a code developed by {Cappellari2004} and is a powerful for obtaining the kinematics, but can also be used to obtain the star-formation histories and emission-line features.
It penalises the kinematic derivation and regularises the star-formation histories by smoothing it as long as the fit quality stays constant. 
\textsc{pPXF} was originally developed for the ATLAS$^{3D}$ project on early-type galaxies, which are believed to have steadily varying star-formation histories.

\textsc{STARLIGHT} is a Fortran 77 code written and uses Markov Chain Monte Carlo (MCMC) methods to derive weights, extinction, and kinematics from a spectrum.
Of all the codes \textsc{STARLIGHT}'s approach is the most comparable to \textsc{PyParadise}, with the exceptions that it simultaneously fits the stellar populations and the kinematics, and that there is not option to fit emission lines.

\textsc{Fit3D} is code that is based on the former {lookup-reference}

Probable strong advantage and disadvantage, high degree of freedom to the user, usable as a Python module and can therefore be integrated in other projects, decoupled fitting of kinematics and stellar populations, decoupled fitting of stellar populations and emission lines.
Example with library usage needed, possibly with IPython Notebooks.

\textsc{paradise}, on which \textsc{PyParadise} is based on, is a code written in Fortran by C. J. Walcher, does not apply parametrisation or penalisation of the star-formation histories.
It instead relies upon bootstrapping to obtain realistic uncertainties on the star-formation histories.

Unlike \textsc{GANDALF}, the SSP fitting and the emission line fitting, are two separate procedures.
This is possible because regions with emission lines are masked out from the stellar population fitting.

Throughout the paper we assume a cosmological model with 
$H_0=70\,\mathrm{km}\,\mathrm{s}^{-1}\,\mathrm{Mpc}^{-1}$, $\Omega_{\mathrm{m}}=0.3$, and $\Omega_\Lambda=0.7$.

\section{Installation}
\label{sec:installation}
\textsc{PyParadise} depends on the following software:
\begin{itemize}
    \item Python version 2.7 or 3.x.
        The code might work on earlier Python versions by supplying the argument \codeline{--parallel 1} to every call of \codeline{ParadiseApp.py}, although this is only tested with Python 2.6.
        Python 2.7 or higher is required in order for multi-threading to function properly.
    \item numpy, tested with versions \ldots
    \item scipy, tested with versions \ldots
    \item Either pyfits, tested with versions \ldots, or astropy, tested with versions \ldots.
        In case you are in doubt which package to install, it is recommended to install astropy as the development of pyfits ceased except for possible bug fixes.
    \item PyMC 2.x, tested with versions \ldots.
        At the moment of writing, PyMC 3 is still in beta status and is a big overhaul with respect to PyMC 2.x.
        If an official release of PyMC 3 comes out, we recommend using PyMC 2.x, until further notice.
    \item emcee, tested with versions \ldots
        Can be used instead of PyMC.
\end{itemize}

In order to install \textsc{PyParadise}, you can obtain the code from BitBucket, and install by running the following command from the main directory
\begin{lstlisting}[language=sh]
    python setup.py install
\end{lstlisting}
If you get permissions errors, please consider running the command with superuser-permissions, inside a virtual environment or to change the command to either of the following commands
\begin{lstlisting}[language=sh]
    python setup.py install --user
    python setup.py install --home
\end{lstlisting}
with the first option being the preferred option and the second option requires a correct setup of the \codeline{PYTHONPATH} environment variable.

Note that the installation procedure \textbf{will not} copy the template files to a new location, but will stay in the \fname{templates} directory.
Feel free to move this files to any location, as the location of the template files need to be specified in the parameter-files.

\section{Quick Start}
\label{sec:quickstart}
Copy the example template files from the test directory and adjust the path to the template directory accordingly.
Update the \paramdef{vel_guess} parameter to a velocity close to the final velocity.
Adjust the minimum and maximum values for the allowed ranges for the velocity $v$ and velocity dispersion $\sigma$.
Have a look if the \paramdef{start_wave} and \paramdef{end_wave} contains the wavelength range where you want to perform the fitting.
Similarly, make sure that \paramdef{min_x}, \paramdef{max_x} and \paramdef{min_y}, \paramdef{max_y} cut up your cube/rss file correctly.
You can now run the stellar population fitting by running
\begin{lstlisting}[language=sh]
    ParadiseApp.py <input_file> <prefix> <spec_res> --SSP_par <ssp_par.txt>
\end{lstlisting}
which will produce the following three output files \fname{<prefix>.cont_model.fits}, \fname{<prefix>.cont_res.fits} and \fname{<prefix>.stellar_table.fits}.

\section{Stellar population modelling}
\label{sec:ssp_model}
\subsection{Algorithm}
\label{subsec:ssp_algo}
The kinematics and the stellar populations determination are computationally two different steps and use different methods.
\textsc{PyParadise} iterates through the fitting of the kinematics and stellar populations, e.g. the kinematics of an input spectrum is determined from a certain template spectrum.
These kinematics are then applied to the template library and the best-fitting combination of template spectra are determined.

\subsubsection{Kinematics}
\label{subsec:ssp_kin}
A Markov Chain Monte Carlo (MCMC) lies on the basis of the determination of the kinematics.
The constraints are the provided limits on the values that $v$ and $\sigma$ can take.
A random value for $v$ and $\sigma$ are taken between these limits, followed by a random walk consisting of $n$ steps.
At each step of the random walk, the code evaluates whether the new kinematical values improves or worsens the fit.
If the fit improves, the new values are accepted.

Stationarity is the state when the mean and variance for $v$ and $\sigma$ do not change when proceeding further in the walk.
The first set of steps inside a walk still depend strongly on the initial value and the MCMC is therefore not stationary for those steps.
These values need to be discarded, a process referred to as \textit{burn-in}.

Even with a proper burn-in, there can be some auto-correlation as the $v_i$ and $\sigma_i$ are based on the $v_{i-1}$ and $\sigma_{i-1}$.
When the auto-correlation is weak-to-moderate, the results from a small walk might be affected by auto-correlation.
For moderate-to-strong auto-correlation, the results from a longer walk might still be influenced by auto-correlation.
To counteract the auto-correlation, the chain can be \textit{thinned} by a factor $k$ by only taking every $k^{\rm th}$ sample of the chain.
This then produces a correction value to $v$ and $\sigma$, which are the values for the derived kinematics at this step of the random walk.

From the sample, $(n - m) / k$ points will be used to derive the mean and the errors of the $v$ and the $\sigma$.

\subsubsection{Stellar populations}
\label{subsec:ssp_ssp}
Before the stellar populations are fitted, there are a few preparatory steps.
Due to issues that often arise with in the data reduction and analysis of spectra, the continuum level is uncertain.
To address this problem, \textsc{PyParadise} normalises the input spectra and the template library before fitting.
To normalise the spectra a running mean is applied, with the width specified in the parameters-file.
Another option is to use a $n^{\rm th}$-order polynomial (not implemented yet), which is especially useful for data which has not been flux-calibrated yet.

The stellar population are fitted by applying a non-negative least squares (NNLS) algorithm when comparing a spectrum to the template library of model spectra.
It is non-negative as so far only light-emitting stellar population have been observed.
The template library does not consist of a set of spectra which are independent from each other, and the problem is referred mathematically as an ill-posed problem.
The NNLS therefore does not provide a unique solution and the recovery of the star-formation history using an NNLS algorithm usually only provides at most dozen non-zero weights, e.g. it tells that at most a dozen SSPs construct the full star-formation history of a galaxy.

We therefore provided a bootstrapping procedure, which re-fits the spectra a number of times with a subset of models.
If a contributing SSP in the fiducial fit is real and is removed during the bootstrapping procedure, it should be recovered by the similar SSPs during the bootstrapping.
See the paper for more details on recovery of stellar parameters from simulated galaxy spectra and a validation of bootstrapping to recover trustworthy star-formation histories.

\subsection{Template handling}
\label{subsec:ssp_template}
\subsection{Masking}
\label{subsec:ssp_mask}
There are two different masking processes.
As the spectra are first normalized before the stellar populations are fitted, one needs to mask out regions which can impact the normalisation, hereafter referred as continuum masking.
What one wants to include for continuum masking are strong artefacts/residuals, strong emission lines and strong absorption lines.
The second masking involves features in the spectra which are not due to stellar populations features: this would be the emission lines and any artificial structure in the spectra.
The emission lines and artefacts are therefore masked twice, while the strong absorption lines are only masked out for the continuum determination.

\subsection{Parameters file}
\label{subsec:ssp_param}
\begin{itemize}
    \item \paramdef{tmpldir} (\texttt{string}) is the path to the directory in which the template file is located.
        This might be an absolute path or a relative path.
    \item \paramdef{tmplfile} (\texttt{string}) file contains all the template spectra, and should be formatted in the FITS format.
        There are two HDUs in the template file: the first one contains the spectra and the second one the stellar population parameters.
        The spectra in the first HDU unit are saved in RSS format, with $m$ wavelenegth elements along the first axis and the $n$ spectra along the second axis.
        The first wavelength element is given by the header keyword \textsc{CRVAL1}, the step in wavelength space by the header keyword \textsc{CDELT1}.
        
        In order to determine what the SSP parameters of each spectrum in the template file are, this information is located in the second HDU in table format.
        The $j^{\rm th}$ entry in the table has the SSP pararmeters corresponding to the $j^{\rm th}$ spectrum.
        This table should at least have the following columns:
        \begin{itemize}
            \item \tblcol{Luminosity}, which is the current luminosity of the SSP.
            \item \tblcol{Mass}, which is the current mass of the SSP, usually normalized such that the initial mass of the SSP is $1~M_{\odot}$.
            \item \tblcol{Age}, which is the age of the SSP.
            \item \tblcol{[Fe/H]}, which is the iron abundance of the SSP.
            \item \tblcol{[A/Fe]}, which is the [$\alpha$/Fe] abundance ratio of the SSP.
            \item \tblcol{mass-to-light}, which is used in the conversion of luminosity-weighted stellar population properties to mass-weighted stellar population properties.
        \end{itemize}
    \item \paramdef{tmplinitspec} (\texttt{int}) The initial spectrum to start the fitting with.
        A good guess can lead to a faster convergence, and a lower \paramdef{iterations} value might therefore be sufficient.
    \item \paramdef{vel_guess} (\texttt{float}) 
    \item \paramdef{vel_min}, \paramdef{vel_max}, \paramdef{disp_min}, \paramdef{disp_max} (\texttt{float}).
        The \paramdef{vel_min} and \paramdef{vel_max} define the lower and the upper boundary between which the MCMC calculations will try to converge to the best-fitting $v$.
        The \paramdef{disp_min} and \paramdef{disp_max} values give the boundaries for $\sigma$.
    \item \paramdef{kin_fix} (0 or 1)
        If the \paramdef{kin_fix} parameter is set to 1, the kinematics will not be fitted, but will instead be read from \fname{<prefix>.kin_table.fits}.
        This table is located as the second HDU unit, and should have the values for the kinematics of each spectrum in the columns named \tblcol{vel_fit} and \tblcol{disp_fit}.
    \item \paramdef{excl_fit} (\texttt{string})
        Path to the file which contains the wavelength windows to exclude during the stellar absorption line fitting.
        This may be an absolute path or a relative path.
    \item \paramdef{excl_cont} (\texttt{string})
        Path to the file which contains the wavelength windows to exclude during the determination of the stellar continuum.
        This may be an absolute path or a relative path.
    \item \paramdef{nwidth_norm} (\texttt{int})
        The width of the running mean in pixel space in which the continuum will be determined.
    \item \paramdef{start_wave}, \paramdef{end_wave} (\texttt{float})
        These wavelength limits describe between where the spectra are fitted.
        One can also use the exclusion files for reducing the wavelength space, but the advantage of \paramdef{start_wave} and \paramdef{end_wave} is that the template spectra are truncated between these wavelength limits.
        As it reduces the wavelength range where the spectra are normalizes, it accelerates the computation and the reduced wavelength range reduces the memory footprint of the code.
    \item \paramdef{min_x}, \paramdef{max_x}, \paramdef{min_y}, \paramdef{max_y}
        If the input spectrum is in cube-format, these parameters extract a smaller cube and fit only those spectra.
        So if a cube has the shape (30, 40, 200), there are 30 spectra along the x direction times 40 spectra along the y direction, and each spectrum contains 2000 wavelength elements.
        If \paramdef{min_x} and \paramdef{max_x} are set to 10 and 15, respectively, and \paramdef{min_y} and \paramdef{max_y} are set to 3 and 5 respectively, only the spectra along (10:15, 3:5) are fitted.
    \item \paramdef{iterations} (\texttt{int})
        The fitting of the best (combination of) SSP(s) amd the kinematics are two different computational steps.
        At the first iteration, the \paramdef{tmplinitspec} is used to fit the kinematics.
        The kinematics obtained from this is then applied to the template library and the best-ftting (combination of) SSP(s) are obtained, which gives the SSP parameters and the best-fitted spectrum.
    \item \paramdef{samples} (\texttt{int})
        The total number of random walks performed inside an MCMC chain. 
        The longer the chain, the more reliable becomes the determination of $v$ and $\sigma$. 
    \item \paramdef{burn} (\texttt{int})
        An MCMC chain starts from a random value between \paramdef{vel_min} and \paramdef{vel_max} for $v$ and a random value between \paramdef{disp_min} and \paramdef{disp_max} for $\sigma$.
        As there are a number of random walks required to converge to a value, the first $m$ walks should be discarded for the determination of the $v$ and $\sigma$ values, which is usually referred to as the burn-in length of a chain. \paramdef{burn} corresponds to this $m$ walks that will be discarded.
    \item \paramdef{thin} (\texttt{int})
        A random walk means that the value at a new position $i + 1$ will be correlated to the value at position $i$.
        The \paramdef{thin} parameter therefore takes only every $k^{\rm th}$ value inside a chain, to reduce the correlation between the values inside a chain.
        When plotting a chain, the graphs depict a high degree of correlation, we recommend to put this to a high value.
        The disadvantage of a high-value is that the total sample size needs to increase to obtain a statistically sufficient sample size.
    \item \paramdef{agebins} (\texttt{string})
        Currently implemented and being tested, but not distributed.
    \item \paramdef{Ael} (\texttt{string})
        Not implemented yet.
    \item \paramdef{efflam} (\texttt{string})
        Not implemented yet.
   \item \paramdef{Rv} (\texttt{string})
        Not implemented yet.
\end{itemize}

\subsubsection{Choosing parameter-values}
The key to a succesful run of \textsc{PyParadise} and to reduce computational time is a careful look at the parameter-files.
Choosing an initial template \paramdef{tmplinitspec} which comes close to the final fitted average might lead to a value of \paramdef{iterations} which is a bit lower and therefore makes the calculations faster.
In case you believe that \paramdef{tmplinitspec} does not differ strongly from the input spectrum, a value of 2 for \paramdef{iterations} will likely suffice.
That this value needs be higher than 3 or 4 should be rare.

The value for \paramdef{vel_guess} is important as the template library gets shifted to this velocity and \textit{afterwards} resampled to the instrumental dispersion.
The derived values for $\sigma$ is therefore an additional broadening applied to the instrumental dispersion.
This value can therefore be set to 3000 km/s, instead of the true 4000 km/s without having significant effect on the fitting, but 10000 km/s would become more problematic.
These numbers depend on the situation \textsc{PyParadise} is applied to, with more accurate values becoming more critical towards higher resolution spectroscopy and lower velocity dispersion.

For \paramdef{vel_min}, \paramdef{vel_max}, \paramdef{disp_min}, and \paramdef{disp_max} we recommend values which are not too far away from the real value as it may take longer to converge or the MCMC routine might try for out unrealistic values.
We also do not recommend values that are too close to the real value, since \textsc{PyParadise} assumes that the distribution of $v$ and $\sigma$ in the final MCMC sample is normal distributed.
For $\sigma$, the assumption that distribution of values is normal might break down in the case that the values come close to the minimum values.
We therefore recommend in such a case to put \paramdef{disp_min} to something like $v_{\rm step} / 2 - \sigma_{\sigma}$, with $v_{\rm step}$ the sampling of the spectrum in velocity space and $\sigma_{\sigma}$ the uncertainty of the velocity dispersion.
Insert figure to illustrate this.

In case you are using kinematical information from a third-source, or just simply re-running \textsc{PyParadise} without the computationally expensive kinematic determination, you can fix the kinematics by setting \paramdef{kin_fix} to 1.
We strongly recommend this setup in case \paramdef{iterations} is set to 0.

For the MCMC, we recommend a test run where you save the full chain to the disk (or a plot, needs implementation first though).
An example is illustrated in Figure \dots.
From this test run, one can determine the proper size of the burn-in length should be around 50/70 as the walk is still in the process of converging.
At the moment, we do not have clear cases where thinning a chain would be beneficial.
Some chains do show correlations between the $i^{\rm th}$ and the $(i + 1)^{\rm th}$ steps of the chain, but the combination of multiple chains which run longer than the $\sim 10$ times the largest autocorrelation value should provide enough values.
We therefore recommend letting the value of \paramdef{thin} to be set to 1 or 2.
The autocorrelation value (not outputted yet by \textsc{PyParadise}), should be multiplied by at least 10 to obtain accurate uncertainties on $v$ and $\sigma$.

\section{Emission-line modelling}\label{sec:em}
\textsc{PyParadise} can also fit emission lines after the continuum subtraction has been performed. Hence, emission lines are not fitted simultaneously with the stellar continuum, which has certain advantages and disadvantges. Below we describe the adopted algorithm, available line shapes and how to setup the configuration files for running the emission-line modelling. 

\subsection{Algorithm}\label{subsec:em_algo}
Fitting emission lines is a non-linear process so we employ standard $\chi^2$ minimization algorithms to find the best-fit parameter set for a given model. In a spectrum of a galaxiy usually several emission lines of interest are present so that a typical model is usually of a superposition of several lines across the spectrum. In \textsc{PyParadise} we offer two different but widely adopted non-linear minimization algorithms. The user can chose between a Downhill-Simplex algorithm or the Levenberg-Marqudart algorithms. 

The issue with fast $\chi^2$ algorithms is that they require a good starting point for exploring the $\chi^2$ surface to find the global minimum and not being trapped in a local minimum. Therefore, reasonable starting values are necessary are crucial for successful non-linear emission-line modelling. \textsc{PyParadise} is optimized for automated analysis of a large number of spectra, with a similar redshift, so that an individual tuning of starting values by hand become infeasible. Our strategy is that users only need to make a good guess for the redshift in all spectra and set a pixel window size so that the local spectral shape and integrated flux of each redshifted line can be empirically evaluated to obtain starting values. This procedure works best for isolated lines, but usually provides reasonable starting values for blended lines in most cases to allow a good fit. In very complex cases, however, this approach may not be sufficient.

\subsection{Line profiles}\label{subsec:em_profile}
Only simple \textbf{Gaussian} line profiles are available in the current version of \textsc{PyParadise}. It is planed to expand this further to Gauss-Hermite expansions of lines to take asymmetric line shapes into account. If asymmetric lines are present in the data, currently two separate Gaussian profiles need to be used to capture the asymmetry of lines implicitly.
\subsubsection{Gaussian line}
A Gaussian line is well described by a central wavelength, a line disperion, and an integrated line flux.  In \textsc{PyParadise} the central wavelength is computed from a provided \textit{rest-frame wavelength} for a specific line together with a \textit{redshift expressed in velocity space as $v= c\times z$ in units of km/s}. Also the line disperion is expressed by an \textit{intrinsic velocity dispersion} from which the line dispersion in unit wavelength at a given redshift is computed and enlarged in quadrature by the provided instrumental spectral resolution. Hence, four parameters (of which only three are free parameters to be optimized) need to be provided for each Gaussian component in the model.

A key feature of \textsc{PyParadise} is that coupling redshift, line dispersion and fixing certain line ratios is easy. The user can simply indicate which lines should have the same redshift or intrinsic velocity diserpsion which can significantly reduce the number of free parameters and makes the optimization process more robust. Fixing line ratios is meant for known multiplets of lines with a-prior priors on their ratios.

\subsection{Configuration files for emission line fitting}\label{subsec:em_config}
Using the emission line modelling in \textsc{PyParadise} requires to prepare three configuration files as described below. In most cases they can be re-used for different spectra and purposes.

\subsubsection{Model file}\label{subsec:model_file}
The model file is a ASCII file that defines a list of different line components to be modelled. There is no limit on the number of components. Each component consistents of a block of several lines depending on the specific component as described below. \textbf{A typical parameter file looks like this}:\bigskip\\
\codeline{Gauss: OIII5007}\\
\codeline{restwave 5006.84}\\
\codeline{flux 50.0 1}\\
\codeline{vel 10350.0 1}\\
\codeline{disp 100.0 1}\bigskip\\
\codeline{Gauss: OIII4960}\\
\codeline{restwave 4958.9}\\
\codeline{flux OIII5007:0.33}\\
\codeline{vel OIII5007}\\
\codeline{disp OIII5007}\\

The structure for each component is the same. The first line of each component always starts with a line TYPE: NAME, where TYPE is a component type, i.e. Gauss for Gaussian, and NAME is a unique identifier WITHOUT spaces for the emission line to be fitted. The second line for a Gaussian is the fixed rest-frame wavelength in the same wavelength units as the spectrum to be fitted. What follows are the settings for the three free parameters of a Gaussian to be fitted. The from for each line is PAR VALUE FLAG, where PAR is either flux, vel or disp to discribe the flux, centroid or disperion of the line. Those are starting values for the fitting which may be updated automatically by \textsc{PyParadise}. Flux are in the same units as the spectra multiplied by the wavelength unit, and vel and disp are in units of km/s as described above for the Gaussian component. The FLAG can be set to 1 or 0. In case of 1, the parameter is free to vary and will be fitted, and in case of 0 the parameter will be fixed to the provided value. 

As shown for the second component some parameters of Gaussian can be coupled to other lines in various ways. The syntax for coupling is PAR NAME, where PAR is the parameter to be coupled and NAME is Gaussian component to which this parameter will be tight together. No FLAG parameter is needed in this case as this will be inherreted from the FLAG parameter of the component it is thight with. For the flux parameter it is additionally allowed to add a fixed multiplicative factor for the coupling which is important for known multiplets with fixed ratios. The syntax in this case is then \codeline{flux NAME:FACTOR}, where NAME is again the component the flux value will be replaced and FACTOR is a multiplicative factor for the coupled flux. 

Not all the parameters of a Gaussian need to be coupled and a mixture of free, fixed and coupled parameters are all allowed for a component. However, an important issue is that the component with the primary parameters always need to be listed before the coupled parameters are defined as in the example case above. Since free parameters are usually updated automatically based on the empirical data, it is recommended that the brightest line is set with free parameter in case coupling between several lines is desired. This will provide the best guess value for coupled parameters.

\subsubsection{Wavelength range file}\label{subsec:wavelength_range}
Usually the emission lines are not spread over the full wavelength range of the input spectra. Computation speed and robustness is gained if the modelling of the emission lines is restricted to the wavelength range the emission line actually contribute signnificantly to. Hence, it is highly recommended to provide a file that defines certain wavelength intervals for which the model is actually evluated and a $\chi^2$ value is computed. The syntax of this file is the same as for the wavelength masking for the stellar continuum fitting, but only wavelength ranges defined in the rest-frame will be used. For the simple case of fitting the [OIII] double as defined in the Model file example above the wavelength range file would simply look like:\bigskip\\
\codeline{[rest-frame]}
\codeline{4930 5030}\bigskip\\

This means that only the wavelength range from 4930 to 5030 (in Angstrom) would be fitted and the rest of the spectrum would be ignored. Of course, the unit of the wavelength ranges needs to match with that of the spectrum and adjusted as necessary.


\subsubsection{Parameter file}
All the setups for the emission line fitting will be combined in a parameter file which is provided to \textsc{PyParadise} at the start-up call. A typical example file looks like this:\bigskip\\
\codeline{eCompFile  modelfile.txt !name of the line parameter file, None if no emission lines to be fitted}\\
\codeline{vel\_guess   10300.0 !rough velocity guess for the object in km/s as cz (float)}\\
\codeline{line\_fit\_region lines.fit !Wavelength regions considered during the fitting (string)}\\
\codeline{efit\_method leastsq !method for line parameter fitting (leastsq/simplex)}\\
\codeline{efit\_ftol  1e-8 !ftol convergence parameter}\\
\codeline{efit\_xtol 1e-8 !xtol convergence parameter}\\
\codeline{eguess\_window 20 !size of the spectral window in pixel to estimate initial guesses for each line}\\
\codeline{min\_x 1 !Minimum x dimension}\\
\codeline{max\_x 500 !Maximum x dimension}\\
\codeline{min\_y 1 !Minimum y dimension}\\
\codeline{max\_y 500 !Maximum y dimension}\bigskip\\
Each line follows the usual convention for \textsc{PyParadise} parameter files, i.e. PAR VALUE !Description. The detailed description of the necessary parameters in this files are\\
\paramdef{eCompFile} (\texttt{string})  Full path to the model file which is in ASCII format as described in Sect.~\ref{subsec:model_file} above.\\
\paramdef{vel\_guess} (\texttt{float})  Best guess of mean redshift $z$ of all spectra to model expressed in velocity (km/s) via $c\times z$.\\
\paramdef{line\_fit\_region} (\texttt{string}) Full path to the wavelength range file in ASCII format as described in Sect.~\ref{subsec:wavelength_range} above.\\
\paramdef{efit\_method} (\texttt{string}) Method to find the minimal $\chi^2$ value. Only two options available now, either leastsq or simplex for the Levenberg-Marquardt or Downhill-simplex method, respectively.\\
\paramdef{efit\_ftol} (\texttt{float}) and \paramdef{efit\_ftol} (\texttt{float}) are the relative errors desired for the $\chi^2$ and the in the approximate solution, respectively, which set stopping limits for the $\chi^2$ minimization alrogithms. Do not change numbers if unsure.\\
\paramdef{eguess\_window} (\texttt{integer}) Based on the mean redshift defined by \paramdef{vel\_guess} and rest-frame wavelength of all lines defined in the model file, empirial measurements on line shape such as integrated flux, line centroid and line shape will be performe from all pixels within the given wavelength window in pixels. If this number is set to 0, no empirical estimates for the starting values will be done and the values in the model file will be used instead (CHECK IF TRUE!?).\\
\paramdef{min\_x} (\texttt{integer})  till \paramdef{max\_y} (\texttt{integer}) define limits for spectra to be modelled in spatial domain. Indexing starts with 1. The syntax is the same as for the continuum modelling as described in Sect.~\ref{}.\\

\section{Bootstrap errors}
\label{sec:bootstrap}
\subsection{The importance of error estimation}
\label{subsec:bootstrap_info}
The SSP fitting only provides the mean values for the SSP parameters and no errors.
Bootstrap provides a technique to obtain the errors on the SSP parameters.
As we already obtained the errors on the kinematics when we fit the spectra for the first time, the kinematics stay constant and the bootstrapping is therefore performed relatively fast.
The \textsc{bootstraps} command-line argument gives the number of bootstraps that should be performed.
The bootstrapping is only performed together when the \textsc{modkeep} command-line argument is provided too.
During the bootstrapping a random number of spectra is discarded.
The command-line argument \textsc{modkeep} gives the percentage of spectra that will be kept during each bootstrapping step.
It is recommended to keep this value relatively high, e.g. 80$\%$.

\section{Data input formats}
\label{sec:data}
Like any software package \textsc{PyParadise} also requires that data is provided in certain pre-defined formats to run successfully.
Failure of the program may often be related to invalid input. 
Therefore, users are strongly encourage to read this section carefully before using this software for their analysis. 
\subsection{Spectral data}
\label{subsec:data_multidim}
\textsc{PyParadise} can be feed with three basic types of spectral data.
These are 1) a single spectrum, 2) row-stacked spectra (hereafter RSS), and 3) 3D data cubes.
The minimum information they all contain are the wavelength grid and the corresponding flux densities that makes up the spectrum.
Optimal but often important information are the associated errors and known bad pixels.
All three types have different technical specifications to store those data.
They are automatically identified by \textsc{PyParadise} if the format is consistent with the specifications.

\subsubsection{General structure of spectral FITS files}
There is no standard for storing spectral information in FITS files.
The format used by \textsc{PyParadise} is described in the section, and we note to the users of the Fortran code of \textsc{paradise} that the format is slightly different.
In this part, the general description of FITS files is given and the variation due to the different formats are given in \S\ref{subsec:data_single} for FITS files containing single spectra, in \S\ref{subsec:data_rss} for RSS files and in \S\ref{subsec:data_cube} for cubes.
\S\ref{subsec:data_single} also contains information regarding the structure of a spectrum which is stored in FITS table format or ASCII format.

A FITS file consists of several HDUs, Header Data Units.
The spectra (\headerval{DATA}), the errors (\headerval{ERROR}), and the pixel mask (\headerval{BADPIX}) are stored in different HDUs.

The \headerval{DATA} HDU should be the first HDU and include a few header values regarding the wavelength information.
Any other information is not ignored (\headerval{DC-FLAG} still needs to be made consistent).
In the header of \headerval{ERROR}, the header keyword \headerkw{EXTNAME} should be set to \headerval{ERROR}.
Similarly, in the header of \headerval{BADPIX}, the header keyword \headerkw{EXTNAME} should be set to \headerval{BADPIX}.
No other header information will be read in from \headerval{ERROR} and \headerval{BADPIX}.

\subsubsection{RSS files}
\label{subsec:data_rss}
A RSS file is the primary output of a multi-fiber spectrograph, either a MOS or IFU system.
It is a simple stack of spectra that share a \textbf{common} wavelength grid.
The stored data should have the structure of $m \times n$ pixels, where $m$ is the number of wavelength elements and $n$ the number of spectra.
The wavelength information of RSS files is given by the header keywords \headerkw{CRVAL1}, first wavelength element, and \headerkw{CDELT1}, the wavelength difference between the $i^{\rm th}$ and $(i+1)^{\rm th}$) pixel, in units of \r{A}ngstrom per pixel.
As any standard and valid FITS file, it should contain the header keyword \headerkw{NAXIS1}, which equals $m$, and it should contain the header keyword \headerkw{NAXIS2}, which is equal to $n$.

\subsubsection{Data cubes}
\label{subsec:data_cube}
A cube is the primary output of an integral field unit (IFU), and can also be the end-products of radio telescope observations.
It is a combination of an image, with at every spatial position (referred as spaxel) a spectrum.
The stored data should have the structure of $x \times y \times z$ pixels, where $x$ and $y$ span the spatial space and $z$ is the number of wavelength elements.
The wavelength information of a cube is given by the header keywords \headerkw{CRVAL3}, first wavelength element, and \headerkw{CDELT3}, the wavelength difference between the $i^{\rm th}$ and $(i+1)^{\rm th}$) pixel, in units of \r{A}ngstrom per pixel.
As any standard and valid FITS file, it should contain the header keyword \headerkw{NAXIS3}, which equals $z$, and it should contain the header keyword \headerkw{NAXIS1} and \headerkw{NAXIS2}, which equal to $x$ and $y$ respectively.

\subsubsection{Single Spectrum}
\label{subsec:data_single}
There are various possible format to store a single spectrum.
To facilitate the usage of PyParadise we implement a few common formats, the SDSS-DR7-like fits image format, the SDSS-DR10-like table format, and an ascii table format.
They are widely distributed in the community.
Since PyParadise does not require all the informations that are stored in SDSS spectra, a user can easily create a confrom spectrum that can be used with PyParadise only with the minimum content as described below.

\subsection{Template spectra}
\label{subsec:data_templates}
The template spectra are a collection of spectra that should ideally be sufficient to describe the spectrum to be modelled as a linear superposition modolu the kinematic convolution kernel.
There for the data format is similar to the RSS case for the input data.
The big difference is that these data are usually theoretical or semi-empirical derived spectra.
Thus, we assume them to be noise free and do not contain masked pixel.
In addition, each template spectrum has associated additional information like stellar population age, luminosity, and metallicity that needs to be captures. 


\section{Running PyParadise}
\label{sec:usage}
To run \textsc{PyParadise}, one needs besides the input spectrum \fname{<input_file>} a few configuration files.
For the stellar populations one parameter file \fname{<ssp_par.txt>} and two mask files are needed, and optionally a file describing the binning of stellar population parameters in different age ranges.
To run \textsc{PyParadise} for extracting the stellar populations, execute the following command:
\begin{lstlisting}[language=sh]
    ParadiseApp.py <input_file> <prefix> <spec_res> --SSP_par <ssp_par.txt>
\end{lstlisting}
where \codeline{<prefix>} is the prefix that will be placed before the names of the output files, and \codeline{<spec_res>} is the spectral resolution of \fname{<input_file>} in FWHM.

The emission line fitting requires the stellar population fitting to be performed first, after which the determination of the emission lines can be performed.
For the emission line fitting, there is a parameter file \codeline{<line_par.txt>}, a mask file, and a file containing the emission line properties.
To run \textsc{PyParadise} for extracting the emission-line properties, execute the following command:
\begin{lstlisting}[language=sh]
    ParadiseApp.py <input_file> <prefix> <spec_res> --line_par <line_par.txt>
\end{lstlisting}
with \codeline{<prefix>} should be the same prefix as used for the SSP derivation.

To obtain errors for the stellar population parameters and the emission line parameters, another run of \textsc{PyParadise} is required, where $m$ bootstrap runs will be performed.
During each bootstrap run, a certain percentage $p$ of the template library will be used to obtain the best-fitting stellar population parameters.
If bootstrapping for emission lines is requested, the best-fitted model at each bootstrap run will be subtracted from \fname{<input file>} and the emission lines will be fitted to this spectrum.
The bootstrapping will thus provide $m$ different values for the stellar population and emission line parameters, from which a bootstrapped mean value and a standard deviation are derived.
To perform the bootstrapping for the stellar population properties only or for the stellar population and emission line properties, run the following command, respectively
\begin{lstlisting}
    ParadiseApp.py <input_file> <prefix> <spec_res> --SSP_par <ssp_par.txt> --bootstraps <m> --modkeep <p>
    ParadiseApp.py <input_file> <prefix> <spec_res> --SSP_par <ssp_par.txt> --line_par <line_par.txt> --bootstraps <m> --modkeep <p>
\end{lstlisting}

We do not recommend interrupting \textsc{PyParadise} call by hitting Ctrl+C on Linux systems, as it might happen that the program will not be interrupted.

\section{Output files}
\label{subsec:usage_output}
\subsection{Stellar population results}
\label{subsec:usage_output_ssp}
There are three files produced as result of the stellar population fitting: \fname{<prefix>.cont_model.fits} (hereafter abbreviated with \fname{<cont model>}), \fname{<prefix>.cont_res.fits} (hereafter abbreviated as \fname{<cont res>}) and \fname{<prefix>.stellar_table.fits} (hereafter abbreviated as \fname{<stellar table>}).
The file \fname{<cont model>} contains the best-fitted stellar population model spectra.
The file \fname{<cont res>} contains the residual spectra, constructed by subtracting the model spectra from the input spectra.

The header information from the \texttt{<input file>} will be copied into the first HDUs of the output files \fname{<cont model>} and \fname{<cont res>}.
A possible exception is the wavelength information and the size of the data cubes depending on the values of the \paramdef{start_wave} and \paramdef{end_wave} parameters.
If the \paramdef{start_wave} value is higher than the first wavelength element, then the first HDU will have a different value of \textsc{CRVAL1} compared to the \texttt{<input file>}.
In the case of single spectrum file or RSS, the value of \textsc{NAXIS1} and in the case of cubes the value of \textsc{NAXIS3} will change too if \paramdef{start_wave} and \paramdef{end_wave} are different from the limits in \texttt{<input file>}.
A change in the header values is linked to a different size of the data in the output files.

In case the fitting of a spectrum failed, the files will contain zero as the model spectrum and therefore the original spectrum as residual.
One can check if the fitting of a spectrum failed by checking if the spectral index is missing from \fname{<stellar table>}.

The kinematic and stellar population parameters, together with the parameters describing the quality of the fit, are saved in the \fname{<stellar table>}.
\begin{itemize}
    \item \tblcol{fiber} (\texttt{int}).
        This parameter is only written if the \texttt{<input file>} is a single spectrum or a RSS file.
        Values might be missing from this column in the case of a fiber where the fitting failed.
    \item \tblcol{xcor}, \tblcol{ycor} (\texttt{int}).
        These parameters are only written if the \texttt{<input file>} is a cube file.
        Values might be missing from this column in the case of a fiber where the fitting failed.
    \item \tblcol{vel_fit}, \tblcol{vel_fit_err} (\texttt{float})
        The fitted value and the corresponding error for $v$ in km/s.
    \item \tblcol{Rvel}
        The Gelman-Rubin convergence value for $v$.
        For multiple runs of the MCMC chains, this value describes how close the derived $v$'s are to each other.
        The closer this value is to 1, the better the convergence of multiple chains to the same value and the more confident one can be that derived value is correct.
    \item \tblcol{disp_fit}, \tblcol{disp_fit_err} (\textsc{float})
        The fitted value and the corresponding error for $\sigma$ in km/s.
    \item \tblcol{Rdisp}
        Similar to \tblcol{Rval}, but for $\sigma$.
    \item \tblcol{chi2} (\texttt{float}).
        The $\chi^2$ value of the fit.
        Use this value to evaluate the quality of the fit.
    \item \tblcol{dof} (\textsc{int})
        Not implemented yet, but should be, as it makes diagnostic much easier.
    \item \tblcol{base_coeff} (\texttt{array of floats})
        This array gives the coefficients of the template library, and the size of the array equals the number of template spectra.
        One can reproduce the model spectrum in \fname{<cont model>}, by multiplying the \tblcol{base_coeff} with the template spectra, and correcting for the normalisation effects and \tblcol{lum_coeff_frac_total}.
    \item \tblcol{lum_coeff_frac_total} (\texttt{float}).
        When fitting a normalised spectrum, in the ideal case the sum of \tblcol{base_coeff} value will equal 1.
        Due to noise and template mismatches, this value in normal cases is close to 1.
    \item \tblcol{lum_<ssp_prop>_total} (\texttt{float}).
        The luminosity-weighted stellar population average of a stellar population property, where \tblcol{<ssp_prop>} can be \tblcol{age}, \tblcol{[Fe/H]}, \tblcol{[A/Fe]}.
    \item \tblcol{mass_coeff_frac_total} (\texttt{float}).
        When fitting a normalised spectrum, in the ideal case the sum of \tblcol{base_coeff} value will equal 1.
        Due to noise and template mismatches, this value in normal cases is close to 1.
        The mass-weighting approach will make this value slightly different from \tblcol{lum_coeff_frac_total}.
    \item \tblcol{mass_<ssp_prop>_total} (\texttt{float}).
        The mass-weighted stellar population average of a stellar population property, where \tblcol{<ssp_prop>} can be \tblcol{age}, \tblcol{[Fe/H]}, \tblcol{[A/Fe]}.
        To obtain the mass-weighted properties, a conversion is performed based on the \tblcol{mass-to-light}-values inside the template library.
    \item More items will be added here, once the agebins file implementation is completed.
\end{itemize}
A number of additional parameters are saved in the file \fname{<prefix>.stellar_table.fits} during the bootstrapping procedure. The columns with the appendix \tblcol{*_err} contain the standard deviation of the corresponding value and the columns with the appendix \tblcol{*_btmean} contain the arithmetic mean of the corresponding value.
\begin{itemize}
    \item \tblcol{lum_coeff_frac_total_err} (\texttt{float}).
     \item \tblcol{lum_coeff_frac_total_btmean} (\texttt{float}).
    \item \tblcol{lum_<ssp_prop>_total_err} (\texttt{float}).
    \item \tblcol{lum_<ssp_prop>_total_btmean} (\texttt{float}).
    \item \tblcol{mass_coeff_frac_total_err} (\texttt{float}).
     \item \tblcol{mass_coeff_frac_total_btmean} (\texttt{float}).
    \item \tblcol{mass_<ssp_prop>_total_err} (\texttt{float}).
     \item \tblcol{mass_<ssp_prop>_total_btmean} (\texttt{float}).
\end{itemize}

\subsection{Emission-line results}
\label{subsec:usage_output_line}
The emission-line fitting produces three files in a similar structure and style as the SSP fitting: \fname{<prefix>.eline_model.fits} (hereafter referred as \fname{<eline_model>}), \fname{<prefix>.eline_res.fits} (hereafter referred as \fname{<eline_res>}) and \fname{<prefix>.eline_table.fits} (hereafter referred as \fname{<eline_table>}).
The \fname{<eline model>} contains the emission-line fits, without the stellar populations, and should therefore be compared to \fname{<cont res>}.
Subtracting \fname{<eline model>} from \fname{<cont res>} produces the file \fname{<eline res>}.
The structure of the files \fname{<eline model>} and \fname{<eline res>} is the same as \fname{<cont model>} and \fname{<cont res>}.

The parameters of the emission lines are stored in the file \fname{<eline table>}.
The data is located as a FITS table in the second HDU, with the column structure as follows:
\begin{itemize}
    \item \tblcol{fiber}, \tblcol{x_cor}, \tblcol{y_cor} (\texttt{int}).
        Described in \S\ref{subsec:usage_output_ssp}
    \item \tblcol{<line>_flux} (\texttt{float}).
        The integrated flux of the emission line.
        The unit of this quantity depends on the unit of the original input file.
    \item \tblcol{<line>_vel} (\texttt{float}).
        The velocity of the emission line.
    \item \tblcol{<line>_fwhm} (\texttt{float}).
        The FWHM of the emission line.
    \item[] The last three columns are repeated for each emission line specified in the file.
    \item[] The following columns are only produced after the bootstrap run:
    \item \tblcol{<line>_flux_err} (\texttt{float}).
        The error of the integrated flux of the emission line.
    \item \tblcol{<line>_vel_err} (\texttt{float}).
        The error of the velocity of the emission line.
    \item \tblcol{<line>_fwhm_err} (\texttt{float}).
        The error of the FWHM of the emission line.
    \item[] The last three columns are repeated for each emission line specified in the file .
\end{itemize}

\section{Using PyParadise as a python module}
\label{sec:paradise_lib}
\textsc{PyParadise} can be imported and used as a python module.
In the case you simply want to fit a spectrum inside a data cube, print the kinematics and check the model parameters for convergence and $\chi^2$ and plot the spectrum and the best-fit, running the following should suffice.
There is some work required in transforming the library to the same resolution, sampling and wavelength grid as the input data.
\begin{lstlisting}[language=python, showstringspaces=False, frame=tb,
                   keywordstyle=\color{deepblue}, emphstyle=\color{deepred}, stringstyle=\color{deepgreen}]
from Paradise import *
import matplotlib.pyplot as plt

vel_guess = 1400.0  # Guessing velocity is 1400 km/s
z = vel_guess / 300000.
norm_window = 100   # Normalization window of 100 pixels

# Load in a spectrum from a cube
cube = loadCube('spectra.fits')
spec = cube.getSpec(20, 25).normalizeSpec(norm_window)

# Load and prepare the stellar population library
# assumed is that the instrumental dispersion for the cube is 3.0 FWHM angstrom
lib = SSPLibrary('templates.fits') 
lib = lib.matchInstFWHM(3.0, vel_guess)
lib = lib.subWaveLibrary(spec.getWave()[0], spec.getWave()[-1])
lib = lib.resampleWaveStepLinear(spec.getWaveStep(), z)
lib = lib.normalizeBase(norm_window).rebinLogarithmic()

# Fit the data with the model
model = spec.fit_Kin_Lib_simple(templates, nlib_guess=20, vel_min=1200,
                                vel_max=1600, vel_guess=1400, disp_min=50,
                                disp_max=150)
model_spec = model[6]

# Print velocity + error + conversion value which should be close to 1
print('Fitted velocity = {} +/-'.format(model[0], model[1]))
print('Gelman-Rubin parameter for velocity conversion = {}'.format(mode[2]))

# Print velocity dispersion + error + conversion value which should be close to 1
print('Fitted velocity dispersion = {} +/-'.format(model[3], model[4]))
print('The Gelman-Rubin parameter for velocity dispersion conversion = {}'.format(mode[5]))

print('The chi^2 value from the fitting of the stellar population = {}'.format(mode[-1]))

plt.plot(spec.getWave(), spec.getData())                # Plot the data
plt.plot(model_spec.getWave(), model_spec.getData())    # Plot the model
\end{lstlisting}

\section{Guidelines}
\label{sec:guidelines}
\subsection{}
\subsection{Distinguishing bad from good fit results}
\label{subsec:guidelines_bad_vs_good}
A good starting point for determining the quality of the fit is to check whether the \paramdef{chi2} value is close to \paramdef{dof}.
This comparison is only reasonable to perform when the input data includes (realistic) errors.
As usual in statistics, a ratio of \paramdef{chi2} / \paramdef{dof} which is much higher than 1 is a bad fit but a value close to 1 does not imply that the fitted values are close to reality.

There can be several reasons for bad \paramdef{chi2} values, and plotting the input spectrum versus the fitted result can provide a good impression what might be going wrong.
\begin{itemize}
    \item Incorrect positioning of the mask for masking out features for the determination of the continuum.
        An effect of what might happen is depicted in Figure todo.
        This can be easily corrected by adjusting the file \fname{excl.cont}.
        In our example, we will need to adjust \ldots to \ldots.
    \item Besides features in the continuum mask, there might be an incorrect positioning with respect to the moment when the spectrum is fitted.
        This is illustrated in Figure todo.
        This can be easily corrected by adjusting the file \fname{excl.fit}.
        In our example, we will need to adjust \ldots to \ldots.
    \item Incorrect determination of the continuum at the edges.
    \item Very young populations, which are missing from the template library.
\end{itemize}

\subsubsection{Examining the stellar kinematic information}
When either $v$ or $\sigma$ are located close the minimum and/or maximum values provided in the stellar parameters file, the limits can be erroneous and/or something else might be going wrong.
If both the errors (\tblcol{vel_fit_err} or \tblcol{disp_fit_err}) and the Gelman-Rubin parameter (\tblcol{Rvel} or \tblcol{Rdisp}) are low, the issue will likely not be solved by altering the parameters \paramdef{sample}, \paramdef{burn}, and \paramdef{thin}.
This is because multiple random walks converged to the same value and within the chain there is not a lot of variation, indicated by the small errors.

When the errors are small but the values \tblcol{Rvel} and/or \tblcol{Rdisp} differ significantly from 1 (usually should be below 1.1 or 1.2), it is an indication that one or multiple chains converge to a different value for $v$ and/or $\sigma$.
There are two methods that might remedy this.
One method involves making the limits \paramdef{vel_min}, \paramdef{vel_max} and/or \paramdef{disp_min}, \paramdef{disp_max} tighter to reduce the parameter space.
Another method is to increase the values for length of the chain, which has the unfortunate side effect that the computation will take longer and that more memory will be allocated.

\section{Frequently asked questions}
\begin{description}[font=\normalfont\itshape]
    \item [My velocity dispersion value is too high?]
        If the fitted $\sigma$ is too high, check the initial guess for the template spectrum.
        In case your initial guess is a relatively old model, but the expectation is a young stellar population, try adjusting the initial guess to a young stellar population.
        If this does not help immediately, try adjusting the the limits on \paramdef{disp_min} and \paramdef{disp_max}
    \item [\textsc{PyParadise} crashed during a run, what do I do now?]
        First of all, if the error message that \textsc{PyParadise} prints to the screen contains both \codeline{multiprocessing/pool.py} and \codeline{raise self._value}, rerun \textsc{PyParadise} with additional command line argument \codeline{--parallel 1} and also consider adding \codeline{--verbose}.
        The error message indicates that something went wrong when \textsc{PyParadise} entered the part of the code which can run in parallel.
        Unfortunately due to the way that Python propagates the error message, it removes the indication to where actually the program ran into trouble and it is therefore not easy to understand where the code breaks.
\end{description}

\section{Conclusions}
\label{sec:conclusions}

\section*{Acknowledgements}

%\bibliographystyle{bibtex/aa}
%\bibliography{references}

\end{document}
