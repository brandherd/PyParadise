% Version 1: Bernd
 \documentclass[usenatbib,usegraphicx,useAMS]{mn2e}
 \usepackage[T1]{fontenc}
 %\usepackage{txfonts}
 \usepackage{multirow}
 \usepackage{color}
 \usepackage{times}
 \usepackage{amssymb}
 \newcommand{\COzero}{${}^{12}\mathrm{CO}(1-0)$}
 \newcommand{\COtwo}{${}^{12}\mathrm{CO}(2-1)$}
 \newcommand{\Fe}{\textsc{feii}}
 \newcommand{\Ox}{\textsc{[oiii]}}
 \newcommand{\Ni}{\textsc{[nii]}}
 \newcommand{\Su}{\textsc{[sii]}}
 \newcommand{\Hb}{\mbox{H$\beta$}}
 \newcommand{\Ha}{\mbox{H$\alpha$}}
 \newcommand{\HII}{\textsc{hii}}
 \newcommand{\QDeb}{\textsc{qdeblend${}^{\mathrm{3D}}$}}
 \newcommand{\changed}[1]{\textbf{#1}}
%\newcommand{\changed}[1]{#1}
\newcommand{\pasp}{PASP}
\newcommand{\aj}{AJ}
\newcommand{\apj}{ApJ}
\newcommand{\apjs}{ApJS}
\newcommand{\aap}{A\&A}
\newcommand{\aaps}{A\&AS}
\newcommand{\araa}{ARA\&A}
\newcommand{\mnras}{MNRAS}
\newcommand{\apjl}{ApJL}

\newcommand{\myemail}{bhuseman@eso.org}
%\bibpunct{(}{)}{;}{a}{}{,}



\begin{document}

\title{PyParadise User Manual}

\author[Husemann et al.]{B.~Husemann$^{1}$\thanks{ESO fellow, bhuseman@eso.org}, O. Choudhury$^{2}$, J. Walcher$^{2}$\newauthor\\
$^1$European Southern Observatory, Karl-Schwarzschild-Str. 2, 85748 Garching b. M\"unchen, Germany\\
$^2$Leibniz-Institut f\"ur Astrophysik Potsdam, An der Sternwarte 16, 14482 Potsdam, Germany\\
}
\maketitle
\begin{abstract}

\end{abstract}
%\begin{keywords}
%\end{keywords}

\section{Introduction}
Throughout the paper we assume a cosmological model with 
$H_0=70\,\mathrm{km}\,\mathrm{s}^{-1}\,\mathrm{Mpc}^{-1}$, $\Omega_{\mathrm{m}}=0.3$, and $\Omega_\Lambda=0.7$.

\section{Quick Start}

\section{Stellar population modelling}
\subsection{Algorithm}
\subsection{Template handling}
\subsection{Masking}
\subsection{Configuration file}


\section{Emission-line modelling}
\subsection{Algorithm}
\subsection{Line profiles}
\subsection{Configuration file}

\section{Bootstrap errors}
\subsection{The importance of error estimation}

\section{Data input formats}
Like any software package \textsc{PyParadise} also requires that data is provided in certain pre-defined formats to run successfully. Failure of the program may often be related to invalid input. 
Therefore, users are strongly encourage to read this section carefully before using this software for their analysis. 
\subsection{Spectral data}
\textsc{PyParadise} can be feed with three basic types of spectral data. These are 1) a single spectrum, 2) row-stacked spectra (hereafter RSS), and 3) 3D data cubes. All three types have different 
requirements on the data format. They are automatically identified by \textsc{PyParadise} if all of the following specifications are met.
\subsubsection{Single Spectrum}
\subsubsection{RSS files}
\subsubsection{Data cubes}

\subsection{Template spectra}

\section{Running PyParadise}

\section{Output files}
\subsection{Stellar population results}
\subsection{Emission-line results}

\section{Guidelines}
\subsection{}
\subsection{Distuingish bad from good fit results}

\section{Conclusions}



\section{Conclusions}

\section*{Acknowledgements}


%\bibliographystyle{bibtex/aa}
%\bibliography{references}

\end{document}



